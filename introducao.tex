\section{Introdução}
	A comunicação dados é a teoria da ciência da computação que trata da comunicação entre computadores e dispositivos diferentes através de um meio de transmissão comum.  A comunicação de dados necessita de uma infraestrutura de rede, composta de computadores ou hosts, roteadores, switches, hubs, cabeamento estruturado, protocolos de comunicação, etc. Pode ocorrer uma interligação entre redes e a Internet Pública,  e essa infraestrutura segue especificações em normas de padronização, chamadas de RFC (sigla do original em inglês, Request for Comments) \cite{napsol2010}.
	
	A comunicação de dados também ocorre através das redes de telecomunicações, as quais estão em constante aperfeiçoamento para suportar a transmissão de informações com a introdução de novas tecnologias, tanto do lado dos equipamentos da rede, quanto dos meios de transmissão (redes de transporte) e dos sistemas de operação para gerenciamento. 
	
	Uma rede pode ser composta de várias sub-redes, dependentes do tipo de serviço que é provido ao consumidor, como telefonia fixa e  telefonia celular, e isso caracteriza uma rede de telecomunicações 
	Já a internet, na verdade  trata-se de um sistema de redes de computadores interconectadas de proporções mundiais, atingindo muitos países e agregando cerca de mais de 300 milhões de computadores e mais de 400 milhões de usuários (DIZARD, 2000, p. 24). Os computadores pessoais ou redes locais, se conectam a provedores de acesso, que se ligam a redes regionais que, por sua vez, se unem à redes nacionais e internacionais. A informação pode viajar através de todas essas redes até chegar ao seu destino. Aparelhos chamados “roteadores”, instalados em diversos pontos da rede, se encarregam de determinar qual a rota mais adequada.
	
	A internet atual surgiu em meados dos anos 60, como uma ferramenta de comunicação militar alternativa, que resistisse ao conflito nuclear mundial, que não necessitasse de nenhum controle central. Para tanto, as mensagens passariam a ser divididos em “pacotes”, os quais seriam transmitidos com mais rapidez, flexibilidade e tolerância a erros, pois cada computador se torna um “nó” da rede. E sendo apenas um “nó” se este ficar impossibilitado de operar não interferiria no fluxo da rede ( MONTEIRO, 2001).
	
	No Brasil a internet teve seu desenvolvimento no meio acadêmico e científico, e instituições de pesquisa. Somente no ano de 1995 a internet deixou de ser privilégio das universidades e da iniciativa privada para se tornar de acesso público. Desde então o número de provedores que oferecem o serviço e número de usuários que utilizam este recurso aumentam a cada ano.
	
	Diante desse contexto a proposta deste trabalho consiste em estudar o funcionamento da comunicação de dados e implementar uma ferramenta de comunicação de redes. Esse estudo será abordado no capítulo 3 o qual consiste do referencial teórico.  Após o referencial teórico, será abordado no capítulo 4 o desenvolvimento da ferramenta de comunicação, depois no capítulo 5 os resultados e discussões, isto é, como foram aplicados os testes e sobre o desempenho obtido pela ferramenta.  No capítulo 6  o grupo apresentará as considerações finais e o seu entendimento sobre tudo que foi feito. E finalmente as referências para esta pesquisa e o código da implementação desenvolvida como anexo.