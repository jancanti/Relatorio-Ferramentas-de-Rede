\section{Fundamentos de comunicação de dados em rede}
Segundo \nocite{gallo2003comunicaccao}Gallo (2003), uma rede é um conjunto de sistemas que possuem uma forma de comunicação entre si com o objetivo de compartilhar informações.

Como exemplo, podemos citar a rede de telefonia. Cada telefone desta rede possui ligação com qualquer outro telefone - desde que você saiba o seu número. Basta você discar o telefone de uma pessoa e com isso você estabelecerá uma conexão entre o seu telefone e o telefone dela. Os dois aparelhos irão mandar dados uns para os outros - no caso, a conversa entre você e a pessoa do outro lado da linha.

Outro exemplo é a televisão. Os programas de televisão também chegam à você por meio de uma rede. Mas esta possui características bem diferentes das redes de telefonia. Nela não se pode enviar informações para as emissoras de televisão. Somente elas transmitem informações, para você e para milhares de outras pessoas.

	\subsection{História das Tecnologias de Rede}
	Uma rede de computadores envolve a interconexão entre dois ou mais micros, o que permite a troca de dados entre essas unidades e otimiza os recursos de hardware e software. Deve ter regras básicas que garantam o envio seguro de informações. Para ser eficiente, ela precisa que os dados transitem de um computador para outro sem que sofram danos. Também é necessário que a rede seja capaz de determinar corretamente para onde as informações estão indo. Além disso, os computadores interligados tem que poder se identificar uns aos outros e deve existir um modo padronizado de nomear e identificar as partes que a compõem. Atualmente, existem milhões de máquina conectadas à Internet e ela tornou-se tão poderosa que é capaz de transmitir entre computadores todo o tipo de dados como imagens, sons, vídeos, textos escritos e até mesmo programas de computador. \cite{forouzan2009comunicaccao}.
	
	\subsection{Alguns Conceitos de redes de Computadores}
	Redes de Computadores formam uma tecnologia de rede única. Nenhuma outra tecnologia de rede é tão versátil e poderosa como ela. Devido à isso, quando falamos sobre elas, podemos utilizar os seguintes termos listados neste pequeno glossário:
	
	\begin{description}
		\item[Clientes] são computadores que se conectam à um computador central para requisitar que este realize alguma tarefa na qual é especializado.
		\item[Confiabilidade] Em todo o tipo de comunicação à distância, existe a possibilidade de ocorrer um erro na hora de se interpretar os dados. No caso das redes de computadores, isso é algo que pode ocorrer devido à vários motivos como interferência ou o enfraquecimento do sinal com a distância. Para se criar uma rede de computadores confiável, é preciso fazer com que os computadores sejam capazes de detectar erros na transmissão. Uma vez que isso ocorra, pode-se tentar corrigí-los ou então pedir para que os dados sejam retransmitidos.
		\item[Endereço] para que possamos nos comunicar com outro elemento de uma rede, precisamos identificá-lo de alguma forma. Na rede telefônica, por exemplo, para falarmos com outra pessoa, precisamos discar o seu número de telefone - que é único para cada elemento da rede. O mesmo ocorre com a rede de computadores. Cada elemento possui um número único que é reconhecido como seu "Endereço". Quase todos os elementos de uma rede de computadores possuem um endereço. Chamamos o ato de distribuir Endereços para os elementos da rede de Endereçamento.
		\item[Meio] É o ambiente físico usado para conectar membros de uma rede. Por exemplo, no caso dos telefones, o meio é o fio que forma toda a rede telefônica. Computadores podem usar os mais diversos meios, como cabos e ondas de rádio.
		\item[Nó] Não são apenas computadores que podem ser ligados à uma rede de computadores. de fato, as primeiras redes de computadores foram criadas para controlar o caminho percorrido por ligações telefônicas. Existe uma gama muito grande de dispositivos que podem fazer parte deste tipo de rede como terminais, impressoras, repetidoras, pontes, chaves e roteadores. Por causa disso, costumamos chamar cada elemento conectado à uma Rede de Computadores de "Nó".
		\item[Protocolo] Computadores só podem lidar com números binários. Eles só entende 0s e 1s. Por conta disso, é preciso criar algum tipo de alfabeto ou padrão para que possamos nos comunicar com apenas dois tipos de sinais. O nome das regras que os computadores seguem para se comunicar entre si chama-se "Protocolo".
		\item[Roteamento] Rotear significa traçar uma rota. O roteamento é justamente a tarefa de traçar rotas entre os vários elementos de uma rede. Afinal, em uma rede com várias máquinas, é preciso estabelecer qual caminho os dados precisam seguir para que eles não terminem indo parar na máquina errada.
		\item[Segurança] É comum que informações sigilosas sejam trocadas em uma rede. Por causa disso, existem muitas pessoas que podem tentar interceptar os dados. Para isso, pode-se utilizar várias estratégias para aumentar a segurança de uma rede como criptografar os dados, por exemplo.
		\item[Servidor] Um Servidor é uma máquina que costuma ser freqüentemente acessada por outras para que ela realize algum tipo de tarefa.
	\end{description}

	\subsection{Classificação de Redes}
	Quanto ao Tamanho da Rede
		\begin{description}
			\item[PAN] Esta é uma sigla para Personal Area Network(Rede de Área Pessoal). Chamam-se assim pequenas redes domésticas de computadores. O alcance destas redes normalmente é o de alguns poucos metros.
			\item[LAN] 
		\end{description}





