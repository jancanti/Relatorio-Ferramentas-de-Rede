\section{Fundamentos de comunicação de dados em rede}
Segundo \nocite{gallo2003comunicaccao}Gallo (2003), o conceito de comunicação de dados na rede é transmissão eletrônica de dados
de um sistema para outro, ou seja, a maneira pela qual computadores trocam informações
uns com os outros. Um título que comumente transmite um significado similar é
“comunicação de dados”. Embora esses dados sejam de fácil troca, algumas pessoas restringem
para incluir somente fatos básicos, e usam o termo informação para sugerir a
organização desses fatos numa forma que tenha significado para seres humanos.

Já \citet{forouzan2009comunicaccao} diz isso isso e aquilo.

\lipsum[1]

	\subsection{Lan}
	\lipsum[1]
	
	\subsection{Wan}
	\lipsum[1]